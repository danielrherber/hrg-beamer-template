%------------------------------
\begin{myslidefragile}{b}{Footnote Commands}

Use\parnote{This is the first one} \lstinline{\parnote} and\parnote{This is the second one} \lstinline{\parnotefull} for\parnotefull{This is the third one that takes up the rest of the line} footnotes\parnotefull{This is the forth one that takes up the rest of the line}\parnote{This is the fifth one}

Use the \lstinline{b} slide option when you have footnotes

\end{myslidefragile}

%------------------------------
\begin{myslidefragile}{b}{URL Commands}

\lstinline{\urlfull} with an example \urlfull{https://www.engr.colostate.edu/~drherber} and in a footnote\parnote{\urlfull{https://www.engr.colostate.edu/~drherber}}

\lstinline{\urlhttps} with an example \urlhttps{www.engr.colostate.edu/~drherber} and in a footnote\parnotefull{\urlhttps{www.engr.colostate.edu/~drherber}}


\lstinline{\urlvideo} with an example \urlvideo{www.youtube.com/watch?v=N17Od3rY0bA} and in a footnote\parnote{\urlvideo{www.youtube.com/watch?v=N17Od3rY0bA}}

\end{myslidefragile}

%------------------------------
\begin{myslidefragile}{c}{Other Commands}

Use \lstinline{\qedsymbol} for \qedsymbol

Use \lstinline{\myterm} for terms like \myterm{Term} (see next slide and \lstinline{\mytermslides})

Use \lstinline{\myline} for a horizontal dividing line

\myline{0.7\textwidth}

Use \lstinline{\eqrepeat} to repeat the last equation number (good when you want to repeat an equation on the next slide):

\begin{align}
A & = \eqbox{\frac{\pi r^2}{2}}
\end{align}

\eqrepeat

\begin{align}
A & = \eqbox{\frac{\pi r^2}{2}}
\end{align}


\end{myslidefragile}

%------------------------------
\begin{myslide}{b}{Examples of Terms \myterm{Term Title}}

\myterm{Term Text 1} \lipsum[1][1-2] \myterm{Term Text 2} \lipsum[1][3]\parnote{They work in a footnote \myterm{Term Footnote}}

\begin{theorem}[Great Theorem]

\lipsum[1][3] \myterm{Term Theorem Text}

\end{theorem}

\begin{enumerate}
\item \myterm{Term List 1}
\item \myterm{Term List 2}
\end{enumerate}

\begin{itemize}
\item
\begin{myremark}
\myterm{Term Box}
\end{myremark}
\end{itemize}

Doesn't work in equation environments, but you can use inline math such as \myterm{Term $x - \mathcal{L} - \bm{x}$}

\end{myslide}