%----------------------------------------------------------------
% set colors
%----------------------------------------------------------------
% dark/light dependent
\setbeamercolor{title}{fg=textcolor, bg=}
\setbeamercolor{institute}{fg=textcolor, bg=bgcolor}
\setbeamercolor{background canvas}{bg=bgcolor}
\setbeamercolor{normal text}{fg=textcolor}
\setbeamercolor{titlelike}{fg=textcolor,bg=}


\setbeamercolor{block body}{bg=bgcolorlight, fg=textcolor}
\setbeamercolor{block title}{bg=fixedbgcolor, fg=fixedtextcolor}

\setbeamercolor{frametitle}{fg=textcolor, bg=}
\setbeamercolor{section in toc}{fg=textcolor}
\setbeamercolor{subsection in toc}{fg=textcolorlight}



\setbeamercolor{enumerate item}{fg=textcoloremph}
\setbeamercolor{itemize item}{fg=textcoloremph}
\setbeamercolor{local structure}{fg=textcoloremph}
\setbeamercolor{caption name}{fg=textcoloremph}
\setbeamercolor{footlinecolor}{fg=textcoloremph,bg=}

\setbeamercolor{bibliography item}{fg=textcoloremph}
\setbeamercolor{bibliography entry author}{fg=textcolor}
\setbeamercolor{bibliography entry title}{fg=textcoloremph}
% \setbeamercolor{bibliography entry location}{fg=orange}
\setbeamercolor{bibliography entry note}{fg=textcolor}


\setbeamercolor{footnote}{fg=xprimarycolor,bg=}
\setbeamercolor{footnote mark}{fg=xprimarycolor,bg=}

%----------------------------------------------------------------
% formatting changes
%----------------------------------------------------------------
\mode<presentation>
{
	\useinnertheme[]{circles}
	\useoutertheme[footline=authortitle,subsection=false]{miniframes}
	\usecolortheme{whale}
	\setbeamercolor*{palette primary}{use=structure, fg=fixedtextcolor, bg=primarycolor}
	\setbeamercolor*{palette secondary}{use=structure, fg=fixedtextcolor, bg=primarycolor}
	\setbeamercolor*{palette tertiary}{use=structure, fg=fixedtextcolor, bg=primarycolor}
	\usefonttheme[onlymath]{serif}
	% \usefonttheme{serif}
}

% remove navigation bar
\beamertemplatenavigationsymbolsempty

% headline formatting
\makeatletter
\setbeamertemplate{headline}
{%
  \vskip-0.6ex
  \begin{beamercolorbox}{section in head/foot}
  \vskip2pt\insertnavigation{\paperwidth}\vskip2pt
  \end{beamercolorbox}%
}
\makeatother

% footer formatting
\setbeamertemplate{footline}{}

% table of contents formatting
\setbeamertemplate{sections/subsections in toc}[sections numbered]

% equation number formatting
\makeatletter
\def\tagform@#1{\maketag@@@{\textcolor{textcoloremph}{(\ignorespaces#1\unskip\@@italiccorr)}}}
\makeatother


% itemize/enumerate formatting
\setbeamertemplate{itemize/enumerate body begin}{\normalsize}
\setbeamertemplate{itemize/enumerate subbody begin}{\normalsize}
\setbeamertemplate{itemize/enumerate subsubbody begin}{\normalsize}

\setbeamertemplate{itemize item}[circle]
% \setbeamertemplate{itemize subitem}[a]
% \setbeamertemplate{itemize subsubitem}[ball]

% \setbeamertemplate{enumerate item}{\insertenumlabel.}
% \setbeamertemplate{enumerate item}{(\alph{enumi})}
\setbeamertemplate{enumerate item}{%
  \usebeamercolor[bg]{item projected}%
  \color{bg}\normalsize\insertenumlabel.%
}

\setbeamertemplate{enumerate subitem}[default]
\setbeamertemplate{enumerate subsubitem}[default]

% \setbeamertemplate{enumerate subitem}{%
%   \usebeamercolor[bg]{item projected}%
%   \color{bg}\normalsize\insertenumlabeli.%
% }

% https://tex.stackexchange.com/questions/225736
\makeatletter
\newlength{\myitemsep}
\setlength{\myitemsep}{3pt}
\newcommand{\my@beamer@setsep}{%
\ifnum\@itemdepth=1\relax
     \setlength\itemsep{\my@beamer@itemsepi}% separation for first level
     \setlength\topsep{\my@beamer@itemsepi}% separation for first level
   \else
     \ifnum\@itemdepth=2\relax
       \setlength\itemsep{\my@beamer@itemsepii}% separation for second level
       \setlength\topsep{\my@beamer@itemsepii}% separation for second level
     \else
       \ifnum\@itemdepth=3\relax
         \setlength\itemsep{\my@beamer@itemsepiii}% separation for third level
         \setlength\topsep{\my@beamer@itemsepiii}% separation for second level
   \fi\fi\fi}
\newlength{\my@beamer@itemsepi}\setlength{\my@beamer@itemsepi}{\myitemsep}
\newlength{\my@beamer@itemsepii}\setlength{\my@beamer@itemsepii}{\myitemsep}
\newlength{\my@beamer@itemsepiii}\setlength{\my@beamer@itemsepiii}{\myitemsep}
\newcommand\setlistsep[3]{%
    \setlength{\my@beamer@itemsepi}{#1}%
    \setlength{\my@beamer@itemsepii}{#2}%
    \setlength{\my@beamer@itemsepiii}{#3}%
}
\xpatchcmd{\itemize}
  {\def\makelabel}
  {\my@beamer@setsep\def\makelabel}
 {}
 {}

\xpatchcmd{\beamer@enum@}
  {\def\makelabel}
  {\my@beamer@setsep\def\makelabel}
 {}
 {}
\makeatother


%
\setbeamertemplate{frametitle continuation}[from second][(Continued)]

%----------------------------------------------------------------
% reference formatting
%----------------------------------------------------------------

% avoid line breaks in bib
% https://stackoverflow.com/questions/57382286
\setbeamertemplate{bibliography entry article}{}
\setbeamertemplate{bibliography entry title}{}
\setbeamertemplate{bibliography entry location}{}
\setbeamertemplate{bibliography entry note}{}

% set bib icon
% \setbeamertemplate{bibliography item}[text]
\renewcommand{\pgfuseimage}[1]{\raisebox{0.04in}{\faBookmark}}

% disable months and days
\AtEveryBibitem{\clearfield{month}}
\AtEveryBibitem{\clearfield{day}}

% disable pages
\AtEveryBibitem{\clearfield{pages}}

% disable address
\AtEveryBibitem{\clearlist{location}}

% disable in:
\renewbibmacro{in:}{}

% customize doi
% https://tex.stackexchange.com/questions/446145
\DeclareFieldFormat{doi}{%
	\color{textcolorlight}%
	DOI\addcolon\space
	\ifhyperref
	{\href{https://doi.org/#1}{\nolinkurl{#1}}}
	{\nolinkurl{#1}}}
\DeclareFieldFormat{url}{%
	\color{textcolorlight}%
	URL\addcolon\space
	\ifhyperref
	{\href{#1}{\nolinkurl{#1}}}
	{\nolinkurl{#1}}}

% change url style
% https://tex.stackexchange.com/questions/79051
% \renewcommand\UrlFont{\rmfamily}
\renewcommand\UrlFont{\sffamily}

% remove final full stop
% https://tex.stackexchange.com/questions/187443
\renewcommand{\finentrypunct}{}

%
\DeclareNameAlias{sortname}{first-last}



%
% https://tex.stackexchange.com/questions/203764
\renewcommand*{\bibfont}{\small}

%
\RaggedRight
\let\raggedright=\RaggedRight


%
\makeatletter
\def\@minipagerestore{\RaggedRight}
\makeatother

%
\addtobeamertemplate{block begin}{%
    \centering%
    \setlength{\textwidth}{0.85\textwidth}%
}{}


%
\BeforeBeginEnvironment{block}{\begin{adjustbox}{minipage={\linewidth}, center}}
\AfterEndEnvironment{block}{\end{adjustbox}\vspace*{0.05in}}


%----------------------------------------------------------------
% set fonts
%----------------------------------------------------------------
\setbeamerfont{title}{size=\huge}
\setbeamerfont{institute}{size=\footnotesize}
\setbeamerfont{date}{size=\footnotesize}



% \setbeamerfont{footnote}{size=\scriptsize}














%----------------------------------------------------------------
% new and redefined commands (formatting)
%----------------------------------------------------------------

\makeatletter
\newcommand*\fix@beamer@close{%
  \ifnum\beamer@trivlistdepth>0%
    \beamer@closeitem%
  \fi%
}%
\newcommand*\fix@beamer@open{%
  \ifnum\beamer@trivlistdepth>0%
    \gdef\beamer@closeitem{}%
  \fi%
}%


% footnote formatting
\renewcommand{\parnotefmt}[1]{\vfill\onslide<1->\setbeamercovered{} \centerline{\colorbox{bgcolorlight}{\parbox{\textwidth}{\footnotesize\noindent #1}}}}

\makeatother

% footnote marker formatting
\renewcommand{\parnotecusmarkfmt}[1]{{\color{textcoloremph}\textsuperscript{#1}}}

% circled command in mysectionpage
\newcommand*\mycircled[1]{\tikz[baseline=(char.base)]{%
\node[shape=circle,draw,inner sep=2pt, line width=0.5mm] (char) {\large#1};}\vspace{0.2in}\\%
}

% ?
\makeatletter
\let\beamer@writeslidentry@miniframeson=\beamer@writeslidentry
\def\beamer@writeslidentry@miniframesoff{%
  \expandafter\beamer@ifempty\expandafter{\beamer@framestartpage}{}% does not happen normally
  {%else
    % removed \addtocontents commands
    \clearpage\beamer@notesactions%
  }
}
\newcommand*{\miniframeson}{\let\beamer@writeslidentry=\beamer@writeslidentry@miniframeson}
\newcommand*{\miniframesoff}{\let\beamer@writeslidentry=\beamer@writeslidentry@miniframesoff}
\makeatother


% left align miniframes on top of slide and place slide number
\newlength\sectionsep
\setlength\sectionsep{6pt}% default 1.875ex plus 1fill

\makeatletter
\def\sectionentry#1#2#3#4#5{% section number, section title, page
  \ifnum#5=\c@part%
  \beamer@section@set@min@width
  \box\beamer@sectionbox\hskip\sectionsep%original: 1.875ex plus 1fill%
  \beamer@xpos=0\relax%
  \beamer@ypos=1\relax%
  \setbox\beamer@sectionbox=
  \hbox{\def\insertsectionhead{#2}%
    \def\insertsectionheadnumber{#1}%
    \def\insertpartheadnumber{#5}%
    {%
      \usebeamerfont{section in head/foot}\usebeamercolor[fg]{section in head/foot}%
      \ifnum\c@section=#1%
        \hyperlink{Navigation#3}{{\usebeamertemplate{section in head/foot}}}%
      \else%
        \hyperlink{Navigation#3}{{\usebeamertemplate{section in head/foot shaded}}}%
      \fi}%
  }%
  \ht\beamer@sectionbox=1.875ex%
  \dp\beamer@sectionbox=0.75ex%
  \fi\ignorespaces}
\def\insertnavigation#1{%
  \vbox{{%
    \usebeamerfont{section in head/foot}\usebeamercolor[fg]{section in head/foot}%
    \beamer@xpos=0\relax%
    \beamer@ypos=1\relax%
    \hbox to #1{\hskip\sectionsep\setbox\beamer@sectionbox=\hbox{\kern1sp}%
      \ht\beamer@sectionbox=1.875ex%
      \dp\beamer@sectionbox=0.75ex%
        \hskip-\sectionsep% original -1.875ex plus 1fill
        \global\beamer@section@min@dim\z@
        \dohead%\hskip\sectionsep%
        \beamer@section@set@min@width
      \box\beamer@sectionbox\hfill\hfill%
      \raisebox{-2pt}{\insertframenumber}\hskip\sectionsep%
      %\hfill%\hskip.3cm
      }%
  }}}
\makeatother

%----------------------------------------------------------------
% new and redefined commands (new things)
%----------------------------------------------------------------

% line command
\newcommand{\myline}[1]{\begin{center}\textcolor{textcoloremph}{\rule{#1}{0.8pt}}\end{center}}


% \newcommand{\mylinewhite}[1]{\textcolor{white}{\rule{#1}{0.8pt}}}

%
% initialize section counter
\newcounter{seccount}
\setcounter{seccount}{0}

\newcommand{\mysectionpage}[1]{
\stepcounter{seccount}
\miniframesoff
\bgroup
\setbeamercolor{background canvas}{bg=fixedbgcolor}
\begin{frame}[noframenumbering,plain]

\begin{tikzpicture}[overlay, remember picture]
\node[anchor=center] at (current page.center) {
\begin{beamercolorbox}[center]{title}
\vspace*{-0.7in}%
\begin{columns}%
\column{0.90\textwidth}%
\centering%
\Huge%
\color{fixedtextcolor}%
{\mycircled{\theseccount} #1}%
\end{columns}%
\end{beamercolorbox}};
\end{tikzpicture}

\end{frame}
\egroup
\miniframeson

}




%-----------
% animations

\setbeamercovered{transparent}

\setbeamercovered{still covered={\opaqueness<1->{25}},again covered={\opaqueness<1->{25}}}

% \setbeamercovered{still covered={\opaqueness<1->{5}},again covered={\opaqueness<1->{60}}}

% begining of the new definition
% \makeatletter
% \newcommand*\fix@beamer@close{%
%   \ifnum\beamer@trivlistdepth>0%
%     \beamer@closeitem%
%   \fi%
% }%
% \newcommand*\fix@beamer@open{%
%   \ifnum\beamer@trivlistdepth>0%
%     \gdef\beamer@closeitem{}%
%   \fi%
% }%
% \makeatother

\makeatletter

\BeforeBeginEnvironment{itemize}{\fix@beamer@close}
\AfterEndEnvironment{itemize}{\fix@beamer@open}

\BeforeBeginEnvironment{enumerate}{\fix@beamer@close}
\AfterEndEnvironment{enumerate}{\fix@beamer@open}

% \BeforeBeginEnvironment{myremark}{\fix@beamer@close%
% }
% \AfterEndEnvironment{myremark}{\fix@beamer@open%
% }

% \BeforeBeginEnvironment{myimportant}{\fix@beamer@close%
% }
% \AfterEndEnvironment{myimportant}{\fix@beamer@open%
% }

% \BeforeBeginEnvironment{myfuture}{\fix@beamer@close%
% }
% \AfterEndEnvironment{myfuture}{\fix@beamer@open%
% }

% \BeforeBeginEnvironment{myquestion}{\fix@beamer@close%
% }
% \AfterEndEnvironment{myquestion}{\fix@beamer@open%
% }

% \BeforeBeginEnvironment{myoptional}{\fix@beamer@close%
% }
% \AfterEndEnvironment{myoptional}{\fix@beamer@open%
% }

% \BeforeBeginEnvironment{myreview}{\fix@beamer@close%
% }
% \AfterEndEnvironment{myreview}{\fix@beamer@open%
% }

\BeforeBeginEnvironment{figbox}{\fix@beamer@close%
}
\AfterEndEnvironment{figbox}{\fix@beamer@open%
}

\BeforeBeginEnvironment{columns}{\fix@beamer@close%
}
\AfterEndEnvironment{columns}{\fix@beamer@open%
}

\BeforeBeginEnvironment{column}{\fix@beamer@close%
}
\AfterEndEnvironment{column}{\fix@beamer@open%
}

\BeforeBeginEnvironment{mytheorem}{\fix@beamer@close%
}
\AfterEndEnvironment{mytheorem}{\fix@beamer@open%
}

\BeforeBeginEnvironment{extraleftalign}{\fix@beamer@close%
}
\AfterEndEnvironment{extraleftalign}{\fix@beamer@open%
}

\BeforeBeginEnvironment{extraleftalignsub}{\fix@beamer@close%
}
\AfterEndEnvironment{extraleftalignsub}{\fix@beamer@open%
}

\BeforeBeginEnvironment{extraleftalignat}{\fix@beamer@close%
}
\AfterEndEnvironment{extraleftalignat}{\fix@beamer@open%
}

\makeatother
% end of the new definition

\usepackage{fixpauseincludegraphics}

\NewEnviron{extraleftalign}[1]{%
\newline\vspace*{-1pt}%
\setlength{\belowdisplayskip}{5pt}%
\addtocounter{beamerpauses}{-1}\uncover<+->{\makebox[\linewidth]{\hspace{#1}\parbox{\dimexpr\linewidth-#1}{%
\begin{align}%
\BODY
\end{align}
}}}%
}

\NewEnviron{extraleftalignsub}[1]{%
\newline\vspace*{-1pt}%
\setlength{\belowdisplayskip}{5pt}%
\addtocounter{beamerpauses}{-1}\uncover<+->{\makebox[\linewidth]{\hspace{#1}\parbox{\dimexpr\linewidth-#1}{%
\begin{subequations}%
\begin{align}%
\BODY
\end{align}%
\end{subequations}%
}}}%
}

\NewEnviron{extraleftalignat}[2]{%
\newline\vspace*{-1pt}%
\setlength{\belowdisplayskip}{5pt}%
\addtocounter{beamerpauses}{-1}\uncover<+->{\makebox[\linewidth]{\hspace{#1}\parbox{\dimexpr\linewidth-#1}{%
\begin{alignat}{4}%
\BODY
\end{alignat}
}}}%
}

\NewEnviron{extraleftgather}[1]{%
\newline\vspace*{-1pt}%
\setlength{\belowdisplayskip}{5pt}%
\addtocounter{beamerpauses}{-1}\uncover<+->{\makebox[\linewidth]{\hspace{#1}\parbox{\dimexpr\linewidth-#1}{%
\begin{gather}%
\BODY
\end{gather}
}}}%
}

\NewEnviron{extraleftgathersub}[1]{%
\newline\vspace*{-1pt}%
\setlength{\belowdisplayskip}{5pt}%
\addtocounter{beamerpauses}{-1}\uncover<+->{\makebox[\linewidth]{\hspace{#1}\parbox{\dimexpr\linewidth-#1}{%
\begin{subequations}%
\begin{gather}%
\BODY
\end{gather}%
\end{subequations}%
}}}%
}